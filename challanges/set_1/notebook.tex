
% Default to the notebook output style

    


% Inherit from the specified cell style.




    
\documentclass[11pt]{article}

    
    
    \usepackage[T1]{fontenc}
    % Nicer default font (+ math font) than Computer Modern for most use cases
    \usepackage{mathpazo}

    % Basic figure setup, for now with no caption control since it's done
    % automatically by Pandoc (which extracts ![](path) syntax from Markdown).
    \usepackage{graphicx}
    % We will generate all images so they have a width \maxwidth. This means
    % that they will get their normal width if they fit onto the page, but
    % are scaled down if they would overflow the margins.
    \makeatletter
    \def\maxwidth{\ifdim\Gin@nat@width>\linewidth\linewidth
    \else\Gin@nat@width\fi}
    \makeatother
    \let\Oldincludegraphics\includegraphics
    % Set max figure width to be 80% of text width, for now hardcoded.
    \renewcommand{\includegraphics}[1]{\Oldincludegraphics[width=.8\maxwidth]{#1}}
    % Ensure that by default, figures have no caption (until we provide a
    % proper Figure object with a Caption API and a way to capture that
    % in the conversion process - todo).
    \usepackage{caption}
    \DeclareCaptionLabelFormat{nolabel}{}
    \captionsetup{labelformat=nolabel}

    \usepackage{adjustbox} % Used to constrain images to a maximum size 
    \usepackage{xcolor} % Allow colors to be defined
    \usepackage{enumerate} % Needed for markdown enumerations to work
    \usepackage{geometry} % Used to adjust the document margins
    \usepackage{amsmath} % Equations
    \usepackage{amssymb} % Equations
    \usepackage{textcomp} % defines textquotesingle
    % Hack from http://tex.stackexchange.com/a/47451/13684:
    \AtBeginDocument{%
        \def\PYZsq{\textquotesingle}% Upright quotes in Pygmentized code
    }
    \usepackage{upquote} % Upright quotes for verbatim code
    \usepackage{eurosym} % defines \euro
    \usepackage[mathletters]{ucs} % Extended unicode (utf-8) support
    \usepackage[utf8x]{inputenc} % Allow utf-8 characters in the tex document
    \usepackage{fancyvrb} % verbatim replacement that allows latex
    \usepackage{grffile} % extends the file name processing of package graphics 
                         % to support a larger range 
    % The hyperref package gives us a pdf with properly built
    % internal navigation ('pdf bookmarks' for the table of contents,
    % internal cross-reference links, web links for URLs, etc.)
    \usepackage{hyperref}
    \usepackage{longtable} % longtable support required by pandoc >1.10
    \usepackage{booktabs}  % table support for pandoc > 1.12.2
    \usepackage[inline]{enumitem} % IRkernel/repr support (it uses the enumerate* environment)
    \usepackage[normalem]{ulem} % ulem is needed to support strikethroughs (\sout)
                                % normalem makes italics be italics, not underlines
    

    
    
    % Colors for the hyperref package
    \definecolor{urlcolor}{rgb}{0,.145,.698}
    \definecolor{linkcolor}{rgb}{.71,0.21,0.01}
    \definecolor{citecolor}{rgb}{.12,.54,.11}

    % ANSI colors
    \definecolor{ansi-black}{HTML}{3E424D}
    \definecolor{ansi-black-intense}{HTML}{282C36}
    \definecolor{ansi-red}{HTML}{E75C58}
    \definecolor{ansi-red-intense}{HTML}{B22B31}
    \definecolor{ansi-green}{HTML}{00A250}
    \definecolor{ansi-green-intense}{HTML}{007427}
    \definecolor{ansi-yellow}{HTML}{DDB62B}
    \definecolor{ansi-yellow-intense}{HTML}{B27D12}
    \definecolor{ansi-blue}{HTML}{208FFB}
    \definecolor{ansi-blue-intense}{HTML}{0065CA}
    \definecolor{ansi-magenta}{HTML}{D160C4}
    \definecolor{ansi-magenta-intense}{HTML}{A03196}
    \definecolor{ansi-cyan}{HTML}{60C6C8}
    \definecolor{ansi-cyan-intense}{HTML}{258F8F}
    \definecolor{ansi-white}{HTML}{C5C1B4}
    \definecolor{ansi-white-intense}{HTML}{A1A6B2}

    % commands and environments needed by pandoc snippets
    % extracted from the output of `pandoc -s`
    \providecommand{\tightlist}{%
      \setlength{\itemsep}{0pt}\setlength{\parskip}{0pt}}
    \DefineVerbatimEnvironment{Highlighting}{Verbatim}{commandchars=\\\{\}}
    % Add ',fontsize=\small' for more characters per line
    \newenvironment{Shaded}{}{}
    \newcommand{\KeywordTok}[1]{\textcolor[rgb]{0.00,0.44,0.13}{\textbf{{#1}}}}
    \newcommand{\DataTypeTok}[1]{\textcolor[rgb]{0.56,0.13,0.00}{{#1}}}
    \newcommand{\DecValTok}[1]{\textcolor[rgb]{0.25,0.63,0.44}{{#1}}}
    \newcommand{\BaseNTok}[1]{\textcolor[rgb]{0.25,0.63,0.44}{{#1}}}
    \newcommand{\FloatTok}[1]{\textcolor[rgb]{0.25,0.63,0.44}{{#1}}}
    \newcommand{\CharTok}[1]{\textcolor[rgb]{0.25,0.44,0.63}{{#1}}}
    \newcommand{\StringTok}[1]{\textcolor[rgb]{0.25,0.44,0.63}{{#1}}}
    \newcommand{\CommentTok}[1]{\textcolor[rgb]{0.38,0.63,0.69}{\textit{{#1}}}}
    \newcommand{\OtherTok}[1]{\textcolor[rgb]{0.00,0.44,0.13}{{#1}}}
    \newcommand{\AlertTok}[1]{\textcolor[rgb]{1.00,0.00,0.00}{\textbf{{#1}}}}
    \newcommand{\FunctionTok}[1]{\textcolor[rgb]{0.02,0.16,0.49}{{#1}}}
    \newcommand{\RegionMarkerTok}[1]{{#1}}
    \newcommand{\ErrorTok}[1]{\textcolor[rgb]{1.00,0.00,0.00}{\textbf{{#1}}}}
    \newcommand{\NormalTok}[1]{{#1}}
    
    % Additional commands for more recent versions of Pandoc
    \newcommand{\ConstantTok}[1]{\textcolor[rgb]{0.53,0.00,0.00}{{#1}}}
    \newcommand{\SpecialCharTok}[1]{\textcolor[rgb]{0.25,0.44,0.63}{{#1}}}
    \newcommand{\VerbatimStringTok}[1]{\textcolor[rgb]{0.25,0.44,0.63}{{#1}}}
    \newcommand{\SpecialStringTok}[1]{\textcolor[rgb]{0.73,0.40,0.53}{{#1}}}
    \newcommand{\ImportTok}[1]{{#1}}
    \newcommand{\DocumentationTok}[1]{\textcolor[rgb]{0.73,0.13,0.13}{\textit{{#1}}}}
    \newcommand{\AnnotationTok}[1]{\textcolor[rgb]{0.38,0.63,0.69}{\textbf{\textit{{#1}}}}}
    \newcommand{\CommentVarTok}[1]{\textcolor[rgb]{0.38,0.63,0.69}{\textbf{\textit{{#1}}}}}
    \newcommand{\VariableTok}[1]{\textcolor[rgb]{0.10,0.09,0.49}{{#1}}}
    \newcommand{\ControlFlowTok}[1]{\textcolor[rgb]{0.00,0.44,0.13}{\textbf{{#1}}}}
    \newcommand{\OperatorTok}[1]{\textcolor[rgb]{0.40,0.40,0.40}{{#1}}}
    \newcommand{\BuiltInTok}[1]{{#1}}
    \newcommand{\ExtensionTok}[1]{{#1}}
    \newcommand{\PreprocessorTok}[1]{\textcolor[rgb]{0.74,0.48,0.00}{{#1}}}
    \newcommand{\AttributeTok}[1]{\textcolor[rgb]{0.49,0.56,0.16}{{#1}}}
    \newcommand{\InformationTok}[1]{\textcolor[rgb]{0.38,0.63,0.69}{\textbf{\textit{{#1}}}}}
    \newcommand{\WarningTok}[1]{\textcolor[rgb]{0.38,0.63,0.69}{\textbf{\textit{{#1}}}}}
    
    
    % Define a nice break command that doesn't care if a line doesn't already
    % exist.
    \def\br{\hspace*{\fill} \\* }
    % Math Jax compatability definitions
    \def\gt{>}
    \def\lt{<}
    % Document parameters
    \title{challange\_6}
    
    
    

    % Pygments definitions
    
\makeatletter
\def\PY@reset{\let\PY@it=\relax \let\PY@bf=\relax%
    \let\PY@ul=\relax \let\PY@tc=\relax%
    \let\PY@bc=\relax \let\PY@ff=\relax}
\def\PY@tok#1{\csname PY@tok@#1\endcsname}
\def\PY@toks#1+{\ifx\relax#1\empty\else%
    \PY@tok{#1}\expandafter\PY@toks\fi}
\def\PY@do#1{\PY@bc{\PY@tc{\PY@ul{%
    \PY@it{\PY@bf{\PY@ff{#1}}}}}}}
\def\PY#1#2{\PY@reset\PY@toks#1+\relax+\PY@do{#2}}

\expandafter\def\csname PY@tok@w\endcsname{\def\PY@tc##1{\textcolor[rgb]{0.73,0.73,0.73}{##1}}}
\expandafter\def\csname PY@tok@c\endcsname{\let\PY@it=\textit\def\PY@tc##1{\textcolor[rgb]{0.25,0.50,0.50}{##1}}}
\expandafter\def\csname PY@tok@cp\endcsname{\def\PY@tc##1{\textcolor[rgb]{0.74,0.48,0.00}{##1}}}
\expandafter\def\csname PY@tok@k\endcsname{\let\PY@bf=\textbf\def\PY@tc##1{\textcolor[rgb]{0.00,0.50,0.00}{##1}}}
\expandafter\def\csname PY@tok@kp\endcsname{\def\PY@tc##1{\textcolor[rgb]{0.00,0.50,0.00}{##1}}}
\expandafter\def\csname PY@tok@kt\endcsname{\def\PY@tc##1{\textcolor[rgb]{0.69,0.00,0.25}{##1}}}
\expandafter\def\csname PY@tok@o\endcsname{\def\PY@tc##1{\textcolor[rgb]{0.40,0.40,0.40}{##1}}}
\expandafter\def\csname PY@tok@ow\endcsname{\let\PY@bf=\textbf\def\PY@tc##1{\textcolor[rgb]{0.67,0.13,1.00}{##1}}}
\expandafter\def\csname PY@tok@nb\endcsname{\def\PY@tc##1{\textcolor[rgb]{0.00,0.50,0.00}{##1}}}
\expandafter\def\csname PY@tok@nf\endcsname{\def\PY@tc##1{\textcolor[rgb]{0.00,0.00,1.00}{##1}}}
\expandafter\def\csname PY@tok@nc\endcsname{\let\PY@bf=\textbf\def\PY@tc##1{\textcolor[rgb]{0.00,0.00,1.00}{##1}}}
\expandafter\def\csname PY@tok@nn\endcsname{\let\PY@bf=\textbf\def\PY@tc##1{\textcolor[rgb]{0.00,0.00,1.00}{##1}}}
\expandafter\def\csname PY@tok@ne\endcsname{\let\PY@bf=\textbf\def\PY@tc##1{\textcolor[rgb]{0.82,0.25,0.23}{##1}}}
\expandafter\def\csname PY@tok@nv\endcsname{\def\PY@tc##1{\textcolor[rgb]{0.10,0.09,0.49}{##1}}}
\expandafter\def\csname PY@tok@no\endcsname{\def\PY@tc##1{\textcolor[rgb]{0.53,0.00,0.00}{##1}}}
\expandafter\def\csname PY@tok@nl\endcsname{\def\PY@tc##1{\textcolor[rgb]{0.63,0.63,0.00}{##1}}}
\expandafter\def\csname PY@tok@ni\endcsname{\let\PY@bf=\textbf\def\PY@tc##1{\textcolor[rgb]{0.60,0.60,0.60}{##1}}}
\expandafter\def\csname PY@tok@na\endcsname{\def\PY@tc##1{\textcolor[rgb]{0.49,0.56,0.16}{##1}}}
\expandafter\def\csname PY@tok@nt\endcsname{\let\PY@bf=\textbf\def\PY@tc##1{\textcolor[rgb]{0.00,0.50,0.00}{##1}}}
\expandafter\def\csname PY@tok@nd\endcsname{\def\PY@tc##1{\textcolor[rgb]{0.67,0.13,1.00}{##1}}}
\expandafter\def\csname PY@tok@s\endcsname{\def\PY@tc##1{\textcolor[rgb]{0.73,0.13,0.13}{##1}}}
\expandafter\def\csname PY@tok@sd\endcsname{\let\PY@it=\textit\def\PY@tc##1{\textcolor[rgb]{0.73,0.13,0.13}{##1}}}
\expandafter\def\csname PY@tok@si\endcsname{\let\PY@bf=\textbf\def\PY@tc##1{\textcolor[rgb]{0.73,0.40,0.53}{##1}}}
\expandafter\def\csname PY@tok@se\endcsname{\let\PY@bf=\textbf\def\PY@tc##1{\textcolor[rgb]{0.73,0.40,0.13}{##1}}}
\expandafter\def\csname PY@tok@sr\endcsname{\def\PY@tc##1{\textcolor[rgb]{0.73,0.40,0.53}{##1}}}
\expandafter\def\csname PY@tok@ss\endcsname{\def\PY@tc##1{\textcolor[rgb]{0.10,0.09,0.49}{##1}}}
\expandafter\def\csname PY@tok@sx\endcsname{\def\PY@tc##1{\textcolor[rgb]{0.00,0.50,0.00}{##1}}}
\expandafter\def\csname PY@tok@m\endcsname{\def\PY@tc##1{\textcolor[rgb]{0.40,0.40,0.40}{##1}}}
\expandafter\def\csname PY@tok@gh\endcsname{\let\PY@bf=\textbf\def\PY@tc##1{\textcolor[rgb]{0.00,0.00,0.50}{##1}}}
\expandafter\def\csname PY@tok@gu\endcsname{\let\PY@bf=\textbf\def\PY@tc##1{\textcolor[rgb]{0.50,0.00,0.50}{##1}}}
\expandafter\def\csname PY@tok@gd\endcsname{\def\PY@tc##1{\textcolor[rgb]{0.63,0.00,0.00}{##1}}}
\expandafter\def\csname PY@tok@gi\endcsname{\def\PY@tc##1{\textcolor[rgb]{0.00,0.63,0.00}{##1}}}
\expandafter\def\csname PY@tok@gr\endcsname{\def\PY@tc##1{\textcolor[rgb]{1.00,0.00,0.00}{##1}}}
\expandafter\def\csname PY@tok@ge\endcsname{\let\PY@it=\textit}
\expandafter\def\csname PY@tok@gs\endcsname{\let\PY@bf=\textbf}
\expandafter\def\csname PY@tok@gp\endcsname{\let\PY@bf=\textbf\def\PY@tc##1{\textcolor[rgb]{0.00,0.00,0.50}{##1}}}
\expandafter\def\csname PY@tok@go\endcsname{\def\PY@tc##1{\textcolor[rgb]{0.53,0.53,0.53}{##1}}}
\expandafter\def\csname PY@tok@gt\endcsname{\def\PY@tc##1{\textcolor[rgb]{0.00,0.27,0.87}{##1}}}
\expandafter\def\csname PY@tok@err\endcsname{\def\PY@bc##1{\setlength{\fboxsep}{0pt}\fcolorbox[rgb]{1.00,0.00,0.00}{1,1,1}{\strut ##1}}}
\expandafter\def\csname PY@tok@kc\endcsname{\let\PY@bf=\textbf\def\PY@tc##1{\textcolor[rgb]{0.00,0.50,0.00}{##1}}}
\expandafter\def\csname PY@tok@kd\endcsname{\let\PY@bf=\textbf\def\PY@tc##1{\textcolor[rgb]{0.00,0.50,0.00}{##1}}}
\expandafter\def\csname PY@tok@kn\endcsname{\let\PY@bf=\textbf\def\PY@tc##1{\textcolor[rgb]{0.00,0.50,0.00}{##1}}}
\expandafter\def\csname PY@tok@kr\endcsname{\let\PY@bf=\textbf\def\PY@tc##1{\textcolor[rgb]{0.00,0.50,0.00}{##1}}}
\expandafter\def\csname PY@tok@bp\endcsname{\def\PY@tc##1{\textcolor[rgb]{0.00,0.50,0.00}{##1}}}
\expandafter\def\csname PY@tok@fm\endcsname{\def\PY@tc##1{\textcolor[rgb]{0.00,0.00,1.00}{##1}}}
\expandafter\def\csname PY@tok@vc\endcsname{\def\PY@tc##1{\textcolor[rgb]{0.10,0.09,0.49}{##1}}}
\expandafter\def\csname PY@tok@vg\endcsname{\def\PY@tc##1{\textcolor[rgb]{0.10,0.09,0.49}{##1}}}
\expandafter\def\csname PY@tok@vi\endcsname{\def\PY@tc##1{\textcolor[rgb]{0.10,0.09,0.49}{##1}}}
\expandafter\def\csname PY@tok@vm\endcsname{\def\PY@tc##1{\textcolor[rgb]{0.10,0.09,0.49}{##1}}}
\expandafter\def\csname PY@tok@sa\endcsname{\def\PY@tc##1{\textcolor[rgb]{0.73,0.13,0.13}{##1}}}
\expandafter\def\csname PY@tok@sb\endcsname{\def\PY@tc##1{\textcolor[rgb]{0.73,0.13,0.13}{##1}}}
\expandafter\def\csname PY@tok@sc\endcsname{\def\PY@tc##1{\textcolor[rgb]{0.73,0.13,0.13}{##1}}}
\expandafter\def\csname PY@tok@dl\endcsname{\def\PY@tc##1{\textcolor[rgb]{0.73,0.13,0.13}{##1}}}
\expandafter\def\csname PY@tok@s2\endcsname{\def\PY@tc##1{\textcolor[rgb]{0.73,0.13,0.13}{##1}}}
\expandafter\def\csname PY@tok@sh\endcsname{\def\PY@tc##1{\textcolor[rgb]{0.73,0.13,0.13}{##1}}}
\expandafter\def\csname PY@tok@s1\endcsname{\def\PY@tc##1{\textcolor[rgb]{0.73,0.13,0.13}{##1}}}
\expandafter\def\csname PY@tok@mb\endcsname{\def\PY@tc##1{\textcolor[rgb]{0.40,0.40,0.40}{##1}}}
\expandafter\def\csname PY@tok@mf\endcsname{\def\PY@tc##1{\textcolor[rgb]{0.40,0.40,0.40}{##1}}}
\expandafter\def\csname PY@tok@mh\endcsname{\def\PY@tc##1{\textcolor[rgb]{0.40,0.40,0.40}{##1}}}
\expandafter\def\csname PY@tok@mi\endcsname{\def\PY@tc##1{\textcolor[rgb]{0.40,0.40,0.40}{##1}}}
\expandafter\def\csname PY@tok@il\endcsname{\def\PY@tc##1{\textcolor[rgb]{0.40,0.40,0.40}{##1}}}
\expandafter\def\csname PY@tok@mo\endcsname{\def\PY@tc##1{\textcolor[rgb]{0.40,0.40,0.40}{##1}}}
\expandafter\def\csname PY@tok@ch\endcsname{\let\PY@it=\textit\def\PY@tc##1{\textcolor[rgb]{0.25,0.50,0.50}{##1}}}
\expandafter\def\csname PY@tok@cm\endcsname{\let\PY@it=\textit\def\PY@tc##1{\textcolor[rgb]{0.25,0.50,0.50}{##1}}}
\expandafter\def\csname PY@tok@cpf\endcsname{\let\PY@it=\textit\def\PY@tc##1{\textcolor[rgb]{0.25,0.50,0.50}{##1}}}
\expandafter\def\csname PY@tok@c1\endcsname{\let\PY@it=\textit\def\PY@tc##1{\textcolor[rgb]{0.25,0.50,0.50}{##1}}}
\expandafter\def\csname PY@tok@cs\endcsname{\let\PY@it=\textit\def\PY@tc##1{\textcolor[rgb]{0.25,0.50,0.50}{##1}}}

\def\PYZbs{\char`\\}
\def\PYZus{\char`\_}
\def\PYZob{\char`\{}
\def\PYZcb{\char`\}}
\def\PYZca{\char`\^}
\def\PYZam{\char`\&}
\def\PYZlt{\char`\<}
\def\PYZgt{\char`\>}
\def\PYZsh{\char`\#}
\def\PYZpc{\char`\%}
\def\PYZdl{\char`\$}
\def\PYZhy{\char`\-}
\def\PYZsq{\char`\'}
\def\PYZdq{\char`\"}
\def\PYZti{\char`\~}
% for compatibility with earlier versions
\def\PYZat{@}
\def\PYZlb{[}
\def\PYZrb{]}
\makeatother


    % Exact colors from NB
    \definecolor{incolor}{rgb}{0.0, 0.0, 0.5}
    \definecolor{outcolor}{rgb}{0.545, 0.0, 0.0}



    
    % Prevent overflowing lines due to hard-to-break entities
    \sloppy 
    % Setup hyperref package
    \hypersetup{
      breaklinks=true,  % so long urls are correctly broken across lines
      colorlinks=true,
      urlcolor=urlcolor,
      linkcolor=linkcolor,
      citecolor=citecolor,
      }
    % Slightly bigger margins than the latex defaults
    
    \geometry{verbose,tmargin=1in,bmargin=1in,lmargin=1in,rmargin=1in}
    
    

    \begin{document}
    
    
    \maketitle
    
    

    
    \begin{Verbatim}[commandchars=\\\{\}]
{\color{incolor}In [{\color{incolor}1}]:} \PY{k+kn}{import} \PY{n+nn}{os}
        \PY{k+kn}{import} \PY{n+nn}{sys}
        \PY{n}{module\PYZus{}path} \PY{o}{=} \PY{n}{os}\PY{o}{.}\PY{n}{path}\PY{o}{.}\PY{n}{abspath}\PY{p}{(}\PY{n}{os}\PY{o}{.}\PY{n}{path}\PY{o}{.}\PY{n}{join}\PY{p}{(}\PY{l+s+s1}{\PYZsq{}}\PY{l+s+s1}{..}\PY{l+s+s1}{\PYZsq{}}\PY{p}{)}\PY{p}{)}
        \PY{k}{if} \PY{n}{module\PYZus{}path} \PY{o+ow}{not} \PY{o+ow}{in} \PY{n}{sys}\PY{o}{.}\PY{n}{path}\PY{p}{:}
            \PY{n}{sys}\PY{o}{.}\PY{n}{path}\PY{o}{.}\PY{n}{append}\PY{p}{(}\PY{n}{module\PYZus{}path}\PY{p}{)}
            
        \PY{k+kn}{import} \PY{n+nn}{binascii}
        \PY{k+kn}{import} \PY{n+nn}{math}
        \PY{k+kn}{import} \PY{n+nn}{numpy} \PY{k}{as} \PY{n+nn}{np}
        \PY{k+kn}{import} \PY{n+nn}{importlib}
        \PY{k+kn}{from} \PY{n+nn}{libraries} \PY{k}{import} \PY{n}{crypto}
        \PY{k+kn}{from} \PY{n+nn}{libraries} \PY{k}{import} \PY{n}{numeral}
        
        \PY{n}{importlib}\PY{o}{.}\PY{n}{reload}\PY{p}{(}\PY{n}{crypto}\PY{p}{)}
        \PY{n}{importlib}\PY{o}{.}\PY{n}{reload}\PY{p}{(}\PY{n}{numeral}\PY{p}{)}
\end{Verbatim}


\begin{Verbatim}[commandchars=\\\{\}]
{\color{outcolor}Out[{\color{outcolor}1}]:} <module 'libraries.numeral' from 'C:\textbackslash{}\textbackslash{}Users\textbackslash{}\textbackslash{}Oguz\textbackslash{}\textbackslash{}Desktop\textbackslash{}\textbackslash{}Cryptopals Challange\textbackslash{}\textbackslash{}libraries\textbackslash{}\textbackslash{}numeral.py'>
\end{Verbatim}
            
     Loading the Challange 6 Data 

    \begin{Verbatim}[commandchars=\\\{\}]
{\color{incolor}In [{\color{incolor}2}]:} \PY{n}{challange\PYZus{}data\PYZus{}addr} \PY{o}{=} \PY{l+s+s2}{\PYZdq{}}\PY{l+s+s2}{C:}\PY{l+s+se}{\PYZbs{}\PYZbs{}}\PY{l+s+s2}{Users}\PY{l+s+se}{\PYZbs{}\PYZbs{}}\PY{l+s+s2}{Oguz}\PY{l+s+se}{\PYZbs{}\PYZbs{}}\PY{l+s+s2}{Desktop}\PY{l+s+se}{\PYZbs{}\PYZbs{}}\PY{l+s+s2}{Cryptopals Challange}\PY{l+s+se}{\PYZbs{}\PYZbs{}}\PY{l+s+s2}{challanges}\PY{l+s+se}{\PYZbs{}\PYZbs{}}\PY{l+s+s2}{challange\PYZus{}6.txt}\PY{l+s+s2}{\PYZdq{}}
        \PY{n}{data} \PY{o}{=} \PY{n}{np}\PY{o}{.}\PY{n}{loadtxt}\PY{p}{(}\PY{n}{challange\PYZus{}data\PYZus{}addr}\PY{p}{,} \PY{n}{dtype}\PY{o}{=}\PY{l+s+s1}{\PYZsq{}}\PY{l+s+s1}{str}\PY{l+s+s1}{\PYZsq{}}\PY{p}{)}
        \PY{n+nb}{print}\PY{p}{(}\PY{l+s+s2}{\PYZdq{}}\PY{l+s+s2}{Data length is}\PY{l+s+s2}{\PYZdq{}}\PY{p}{,} \PY{n+nb}{len}\PY{p}{(}\PY{n}{data}\PY{p}{)}\PY{p}{)}
        \PY{n}{first\PYZus{}string} \PY{o}{=} \PY{n}{data}\PY{p}{[}\PY{l+m+mi}{0}\PY{p}{]}
        \PY{n}{first\PYZus{}string\PYZus{}len} \PY{o}{=} \PY{n+nb}{len}\PY{p}{(}\PY{n}{first\PYZus{}string}\PY{p}{)}
        \PY{n+nb}{print}\PY{p}{(}\PY{l+s+s2}{\PYZdq{}}\PY{l+s+s2}{String: }\PY{l+s+s2}{\PYZdq{}}\PY{p}{,} \PY{n}{first\PYZus{}string}\PY{p}{)}
        \PY{n+nb}{print}\PY{p}{(}\PY{l+s+s2}{\PYZdq{}}\PY{l+s+s2}{Length: }\PY{l+s+s2}{\PYZdq{}}\PY{p}{,} \PY{n}{first\PYZus{}string\PYZus{}len}\PY{p}{)}
        \PY{n}{first\PYZus{}string}\PY{p}{[}\PY{o}{\PYZhy{}}\PY{l+m+mi}{1}\PY{p}{]}
\end{Verbatim}


    \begin{Verbatim}[commandchars=\\\{\}]
Data length is 64
String:  HUIfTQsPAh9PE048GmllH0kcDk4TAQsHThsBFkU2AB4BSWQgVB0dQzNTTmVS
Length:  60

    \end{Verbatim}

\begin{Verbatim}[commandchars=\\\{\}]
{\color{outcolor}Out[{\color{outcolor}2}]:} 'S'
\end{Verbatim}
            
     Finding the Hamming Distance 

    \begin{Verbatim}[commandchars=\\\{\}]
{\color{incolor}In [{\color{incolor}3}]:} \PY{n}{test\PYZus{}plaintext} \PY{o}{=} \PY{l+s+s2}{\PYZdq{}}\PY{l+s+s2}{this is a test}\PY{l+s+s2}{\PYZdq{}}
        \PY{n}{test\PYZus{}key} \PY{o}{=} \PY{l+s+s2}{\PYZdq{}}\PY{l+s+s2}{wokka wokka!!!}\PY{l+s+s2}{\PYZdq{}}
        
        
        \PY{n}{distance\PYZus{}deneme} \PY{o}{=} \PY{n}{crypto}\PY{o}{.}\PY{n}{Crypto}\PY{p}{(}\PY{n}{plaintext}\PY{o}{=}\PY{n}{test\PYZus{}plaintext}\PY{p}{)}
        
        \PY{n}{distance\PYZus{}deneme}\PY{o}{.}\PY{n}{findHammingDistance}\PY{p}{(}\PY{n}{test\PYZus{}key}\PY{p}{)}
\end{Verbatim}


    \begin{Verbatim}[commandchars=\\\{\}]
0111011101101111011010110110101101100001001000000111011101101111011010110110101101100001001000010010000100100001

    \end{Verbatim}

\begin{Verbatim}[commandchars=\\\{\}]
{\color{outcolor}Out[{\color{outcolor}3}]:} 37
\end{Verbatim}
            
     Finding the Keysize

For each KEYSIZE, take the first KEYSIZE worth of bytes, and the second
KEYSIZE worth of bytes, and find the edit distance between them.
Normalize this result by dividing by KEYSIZE.

The KEYSIZE with the smallest normalized edit distance is probably the
key. You could proceed perhaps with the smallest 2-3 KEYSIZE values. Or
take 4 KEYSIZE blocks instead of 2 and average the distances.

    \begin{Verbatim}[commandchars=\\\{\}]
{\color{incolor}In [{\color{incolor}4}]:} \PY{n}{distance\PYZus{}deneme}\PY{o}{.}\PY{n}{Base2}\PY{p}{(}\PY{p}{)}
        \PY{n}{test\PYZus{}bytes} \PY{o}{=} \PY{n}{distance\PYZus{}deneme}\PY{o}{.}\PY{n}{value}
        \PY{n+nb}{print}\PY{p}{(}\PY{n}{test\PYZus{}bytes}\PY{p}{)}
        
        \PY{n}{keysize} \PY{o}{=} \PY{l+m+mi}{3}
        
        
        \PY{n}{crypto}\PY{o}{.}\PY{n}{Crypto}\PY{o}{.}\PY{n}{findKeysizeScore}\PY{p}{(}\PY{n}{keysize}\PY{p}{,} \PY{n}{test\PYZus{}bytes}\PY{p}{)}
\end{Verbatim}


    \begin{Verbatim}[commandchars=\\\{\}]
0111010001101000011010010111001100100000011010010111001100100000011000010010000001110100011001010111001101110100

    \end{Verbatim}

\begin{Verbatim}[commandchars=\\\{\}]
{\color{outcolor}Out[{\color{outcolor}4}]:} 1.6666666666666667
\end{Verbatim}
            
     Finding the Keysize for the Challange 6 Text

For each KEYSIZE, take the first KEYSIZE worth of bytes, and the second
KEYSIZE worth of bytes, and find the edit distance between them.
Normalize this result by dividing by KEYSIZE.

The KEYSIZE with the smallest normalized edit distance is probably the
key. You could proceed perhaps with the smallest 2-3 KEYSIZE values. Or
take 4 KEYSIZE blocks instead of 2 and average the distances.

    \begin{Verbatim}[commandchars=\\\{\}]
{\color{incolor}In [{\color{incolor}16}]:} \PY{n}{importlib}\PY{o}{.}\PY{n}{reload}\PY{p}{(}\PY{n}{crypto}\PY{p}{)}
         
         
         \PY{n}{data\PYZus{}str} \PY{o}{=} \PY{l+s+s2}{\PYZdq{}}\PY{l+s+s2}{\PYZdq{}} 
         
         \PY{k}{for} \PY{n}{elem} \PY{o+ow}{in} \PY{n}{data}\PY{p}{:}
             \PY{n}{data\PYZus{}str} \PY{o}{+}\PY{o}{=} \PY{n}{elem}
         \PY{n}{data\PYZus{}str} \PY{o}{=} \PY{n}{data\PYZus{}str}\PY{p}{[}\PY{p}{:}\PY{o}{\PYZhy{}}\PY{l+m+mi}{1}\PY{p}{]}
             
         \PY{n}{challange\PYZus{}6\PYZus{}numeral} \PY{o}{=} \PY{n}{numeral}\PY{o}{.}\PY{n}{Numeral}\PY{p}{(}\PY{n}{value}\PY{o}{=}\PY{n}{data\PYZus{}str}\PY{p}{,} \PY{n}{base} \PY{o}{=} \PY{l+m+mi}{64}\PY{p}{)}
         \PY{n}{challange\PYZus{}6\PYZus{}numeral}\PY{o}{.}\PY{n}{Base16}\PY{p}{(}\PY{p}{)}
         
         \PY{n}{challange\PYZus{}6} \PY{o}{=} \PY{n}{crypto}\PY{o}{.}\PY{n}{Crypto}\PY{p}{(}\PY{n}{ciphertext} \PY{o}{=} \PY{n}{challange\PYZus{}6\PYZus{}numeral}\PY{o}{.}\PY{n}{value}\PY{p}{)}
         
         \PY{n}{challange\PYZus{}6\PYZus{}scores} \PY{o}{=} \PY{n}{challange\PYZus{}6}\PY{o}{.}\PY{n}{findKeysize}\PY{p}{(}\PY{p}{)}
         \PY{n+nb}{print}\PY{p}{(}\PY{n}{challange\PYZus{}6\PYZus{}scores}\PY{p}{)}
         
         \PY{n}{N} \PY{o}{=} \PY{l+m+mi}{3}
         \PY{n}{min\PYZus{}score\PYZus{}idx} \PY{o}{=} \PY{n}{np}\PY{o}{.}\PY{n}{array}\PY{p}{(}\PY{n}{challange\PYZus{}6\PYZus{}scores}\PY{p}{)}\PY{o}{.}\PY{n}{argsort}\PY{p}{(}\PY{p}{)}\PY{p}{[}\PY{p}{:}\PY{n}{N}\PY{p}{]}\PY{p}{[}\PY{p}{:}\PY{p}{:}\PY{l+m+mi}{1}\PY{p}{]}
         
         \PY{n}{min\PYZus{}score\PYZus{}idx}
\end{Verbatim}


    \begin{Verbatim}[commandchars=\\\{\}]
[6.         2.5        2.         3.5        1.6        4.16666667
 3.42857143 2.625      3.55555556 3.3        3.09090909 3.5
 2.76923077 3.28571429 2.93333333 3.25       2.88235294 2.94444444
 3.36842105 3.         2.71428571 3.40909091 3.39130435 3.33333333
 3.36       3.61538462 3.40740741 3.53571429 3.27586207 3.5
 3.22580645 3.5        3.39393939 3.44117647 3.42857143 3.52777778
 3.10810811 3.07894737 3.25641026 3.25      ]

    \end{Verbatim}

\begin{Verbatim}[commandchars=\\\{\}]
{\color{outcolor}Out[{\color{outcolor}16}]:} array([4, 2, 1], dtype=int64)
\end{Verbatim}
            
     Finding the Key

Now that you probably know the KEYSIZE: break the ciphertext into blocks
of KEYSIZE length.

Now transpose the blocks: make a block that is the first byte of every
block, and a block that is the second byte of every block, and so on.
Solve each block as if it was single-character XOR. You already have
code to do this.

    \begin{Verbatim}[commandchars=\\\{\}]
{\color{incolor}In [{\color{incolor}57}]:} \PY{n}{keysize\PYZus{}cand} \PY{o}{=} \PY{n}{min\PYZus{}score\PYZus{}idx}\PY{p}{[}\PY{l+m+mi}{0}\PY{p}{]} \PY{o}{+} \PY{l+m+mi}{1}
         \PY{n}{blocksize} \PY{o}{=} \PY{n}{keysize\PYZus{}cand} \PY{o}{*} \PY{l+m+mi}{8}
         \PY{n}{bin\PYZus{}data} \PY{o}{=} \PY{n}{challange\PYZus{}6}\PY{o}{.}\PY{n}{value}
         
         \PY{c+c1}{\PYZsh{} Add 0\PYZsq{}s to the and of the data so that it is evenly divisible to KEYSIZE}
         \PY{n}{bin\PYZus{}data\PYZus{}size} \PY{o}{=} \PY{n+nb}{len}\PY{p}{(}\PY{n}{bin\PYZus{}data}\PY{p}{)}
         \PY{n}{remainder} \PY{o}{=} \PY{n}{bin\PYZus{}data\PYZus{}size} \PY{o}{\PYZpc{}} \PY{n}{keysize\PYZus{}cand}
         \PY{k}{for} \PY{n}{i} \PY{o+ow}{in} \PY{n+nb}{range}\PY{p}{(}\PY{n}{keysize\PYZus{}cand} \PY{o}{\PYZhy{}} \PY{n}{remainder}\PY{p}{)}\PY{p}{:}
             \PY{n}{bin\PYZus{}data} \PY{o}{+}\PY{o}{=} \PY{l+s+s2}{\PYZdq{}}\PY{l+s+s2}{0}\PY{l+s+s2}{\PYZdq{}}
         
         \PY{c+c1}{\PYZsh{} Create number of numerals to hold each block}
         \PY{n}{objs} \PY{o}{=} \PY{p}{[}\PY{n}{numeral}\PY{o}{.}\PY{n}{Numeral}\PY{p}{(}\PY{n}{base}\PY{o}{=}\PY{l+m+mi}{2}\PY{p}{)} \PY{k}{for} \PY{n}{i} \PY{o+ow}{in} \PY{n+nb}{range}\PY{p}{(}\PY{n}{keysize\PYZus{}cand}\PY{p}{)}\PY{p}{]}
         
         \PY{c+c1}{\PYZsh{} Break the ciphertext into blocks of KEYSIZE length.}
         \PY{n}{block\PYZus{}no} \PY{o}{=} \PY{n+nb}{int}\PY{p}{(}\PY{n}{bin\PYZus{}data\PYZus{}size} \PY{o}{/} \PY{n}{blocksize}\PY{p}{)}
         \PY{c+c1}{\PYZsh{} In the first loop divide the data into (Keysize*8) bits}
         \PY{k}{for} \PY{n}{i} \PY{o+ow}{in} \PY{n+nb}{range}\PY{p}{(}\PY{n}{block\PYZus{}no}\PY{p}{)}\PY{p}{:}
             \PY{n}{dummy\PYZus{}block} \PY{o}{=} \PY{n}{bin\PYZus{}data}\PY{p}{[}\PY{p}{(}\PY{n}{i}\PY{o}{*}\PY{n}{blocksize}\PY{p}{)}\PY{p}{:}\PY{p}{(}\PY{n}{i}\PY{o}{+}\PY{l+m+mi}{1}\PY{p}{)}\PY{o}{*}\PY{n}{blocksize}\PY{p}{]}
             \PY{c+c1}{\PYZsh{}In the second loop divide each block, and create a block of first, second, etc bytes}
             \PY{k}{for} \PY{n}{j} \PY{o+ow}{in} \PY{n+nb}{range}\PY{p}{(}\PY{n}{keysize\PYZus{}cand}\PY{p}{)}\PY{p}{:}
                 \PY{n}{dummy\PYZus{}chunk} \PY{o}{=} \PY{n}{dummy\PYZus{}block}\PY{p}{[}\PY{p}{(}\PY{n}{j}\PY{o}{*}\PY{l+m+mi}{8}\PY{p}{)}\PY{p}{:}\PY{p}{(}\PY{n}{j}\PY{o}{+}\PY{l+m+mi}{1}\PY{p}{)}\PY{o}{*}\PY{l+m+mi}{8}\PY{p}{]}
                 \PY{n}{objs}\PY{p}{[}\PY{n}{j}\PY{p}{]}\PY{o}{.}\PY{n}{value} \PY{o}{=} \PY{n}{objs}\PY{p}{[}\PY{n}{j}\PY{p}{]}\PY{o}{.}\PY{n}{value} \PY{o}{+} \PY{n}{dummy\PYZus{}chunk}
             
         
         \PY{c+c1}{\PYZsh{} Solve each block as }
             
\end{Verbatim}


    \begin{Verbatim}[commandchars=\\\{\}]
4601
4601
4601
4601
4601

    \end{Verbatim}


    % Add a bibliography block to the postdoc
    
    
    
    \end{document}
